\documentclass[12pt, twoside]{article}
\usepackage{jmlda}
\newcommand{\hdir}{.}
\usepackage[utf8]{inputenc}
\usepackage[english,russian]{babel}
\usepackage{graphicx}





\begin{document}

\title
    [Шаблон статьи для публикации] % краткое название; не нужно, если полное название влезает в~колонтитул
    {Нейросетевой подход в задаче фазового поиска для изображений с европейского синхротрона}
\author
    { К.\,О.~Вайсер, М.\,С.~Потанин, В.\,В.~Стрижов} % основной список авторов, выводимый в оглавление

\email
    {author@site.ru; co-author@site.ru;  co-author@site.ru}
\thanks
    {Работа выполнена при
     %частичной
     финансовой поддержке РФФИ, проекты \No\ \No 00-00-00000 и 00-00-00001.}
\organization
    {$^1$Организация, адрес; $^2$Организация, адрес}
\abstract
    {
    	Исследуется метод повышения качества разрешения изображений наноразмерных объектов. Рассматривается задача восстановления изображения кокколитов для определения их размеров. В качестве решения используется алгоритм Gerchberg-Saxton, преобразованный c помощью байесовских нейросетей. Для оптимизации результата используются экспертные знания о физических ограничениях. 
	
	
\bigskip
\noindent
\textbf{Ключевые слова}: \emph {преобразование Фурье; байесовская нейросеть; автокодировщик}
}


%данные поля заполняются редакцией журнала
\doi{10.21469/22233792}
\receivedRus{01.01.2017}
\receivedEng{January 01, 2017}

\maketitle


\section{Введение}
Во многих сферах науки широко используется преобразование Фурье для получения изображений из имеющегося спектра исследуемого объекта. Однако эффективному восстановлению реального изображения мешает проблема потерянной фазы. Так как оптические приборы способны измерять только интенсивность излучения, информация о фазе пучка теряется. Данная работа предлагает метод повышения качества изображения спектра для последующего восстановления.

Для получения спектра объекта используется Европейский синхротрон(ESRF). Действительное изображение объекта может быть получено путем применения Фурье преобразования к полученной в результате интерференции электронов картине спектра.

Основной проблемой восстановления изображения является отсутствие записанной мнимой части спектра, так как в синхротроне записывается только действительная часть. Поскольку преобразование Фурье имеет линейную природу, предлагается использовать аппроксимацию сверточной нейронной сетью. Помимо пропущенной фазы, существуют различные дефекты на самом изображении \cite{latychevskaia2018iterative}. 

Широко распротраненным методом восстановления фазы является использование глубоких сверточных нейронных сетей \cite{rivenson2018phase}. Глубокое обучение - это метод машинного обучения, который использует многослойную нейронную сеть для моделирования данных, анализа и принятия решений и показал значительный успех в областях, связанных с большими объемами данных. Авторы \cite{rivenson2018phase} показывают примеры работы данного метода:
\newpage

\begin{figure}[h!]
\centering
\includegraphics[scale = 0.7]{Scheme_conv_net.jpg}
\caption{Схема работы глубокой сверточной нейронной сети}
\end{figure}


После обучения глубокая нейронная сеть способна выводить фазовые и амплитудные изображения объекта без искажений, используя только одну записанную голограмму интенсивности. Глубокая нейронная сеть состоит из сверточных слоев, остаточных блоков и блоков повышенной дискретизации и быстро обрабатывает входное изображение со комплекснозначыми величинами в параллельном  режиме.


Мы будем использовать алгоритм Gerchberg-Saxton \cite{jo2018quantitative}, преобразованный c помощью байесовских нейросетей. Будут использованы экспертные знания о физических ограничения, например положительная плотность потока, для построения функции ошибки.

Вычислительный эксперимент будет проведен с использованием данных о формировании кокколитов \cite{beuvier2019x}. Решение этой задачи позволит приблизиться к пониманию влияния этих организмов на окисление океана.

\bibliographystyle{unsrt}
\bibliography{References}






\end{document}
